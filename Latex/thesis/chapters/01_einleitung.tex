\chapter{Einleitung}
\label{chap:einleitung}

\section{Hintergrund: Die zunehmende Verbreitung von Fake News und deren gesellschaftliche Auswirkungen}
\label{sec:hintergrund}

Fake News sind ein allgegenwärtiges Problem. Als politischer Schachzug bereits im Amerikanischen Bürgerkrieg 1779 genutzt, 
hat der Term auch in der US-Wahl 2016 wieder eine besondere Rolle gespielt \cite{Sharma:2024}.

Fake News aufgeteilt in Satire, Clickbait, Rumor, Stance News, Propaganda, Large Scale Hoaxes \cite{Sharma:2024}-p2

Fake News ist die größte Drohung zu unserer angeblich freien Presse \cite{Sharma:2024}-p3

Fake News als Begriff politischer Kampfbegriff aus Trump-Wahlkämpfen -> stattdessen "Desinformationen" \cite{buerker2022fakenews}
Propaganda arbeitet mit Desinformationen. \cite{buerker2022fakenews}

Nachrichteninhalt: Quelle, Titel-erzielt die Aufmerksamkeit der Lesers, Text-enthält die eigentliche Information, Media (Bilder/Videos)

Sie können die Form von Text, Fotos, Filmen oder Audio annehmen und sind dementsprechend auf jeder Platform auffindbar, 
die die Verbreitung nicht unterbindet. Die momentan populärste Platform zum Teilen der Fake News ist WhatsApp \cite{Ashish2024}.

Support Vector Machines sind die mit am meisten genutzten Kategorisierungs Algorithmen \cite{Sharma:2024}p12
-> Hier gibt es eine Aufzählung an Algorithmen die für das ML Model genutzt werden könnten! (auch deep learning im späteren Teil)

\subsection{Probleme beim Erkennen von Fake News}

Fake News können erst erkannt werden, nachdem diese erstellt und im Internet verbreitet wurden. \cite{Sharma:2024}

\subsection{Klassifizierungen}

Linguistische Features werden von Textmaterial auf verschiedenen Leveln gesammelt, z.B. Buchstaben, Wörter, Sätze 
und Features auf dem Satzlevel (Häufigkeit von Funktionswörtern? und Sätzen) \cite{secrypt17}

Text Tokenisierung \cite{Wagner:2010aa}

Aufteilung der Features in Syntactic, Semantic, Sentiment, Lexical, Style-based \cite{Sharma:2024}p7

\section{Wahl der Nachrichtenportale}
\label{sec:wahl_nachrichtenportale}

\section{Zielsetzung: Entwicklung einer Software zur automatisierten Fake-News-Erkennung}
\label{sec:zielsetzung}

\section{Aufbau der Arbeit}
\label{sec:aufbau}

