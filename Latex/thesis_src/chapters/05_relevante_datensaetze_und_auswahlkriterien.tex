\chapter{Relevante Datensätze und Auswahlkriterien}
\label{chap:relevante_datensaetze_und_auswahlkriterien}

Die in dieser Arbeit entwickelte Anwendung benötigt einen Datenbasis um das Erkennen von Fake News erfolgreich zu lernen.
Hierfür werden verschiedene Datensätze vorgestellt und analysiert.

\section{Vorstellung verfügbarer deutscher Fake-News-Datensätze}
\label{sec:vorstellung_verfuegbare_deutsche_fake_news_datensaetze}

In den Tabellen \ref{tab:deutsche-fake-news-datensaetze} und \ref{tab:deutsche-fake-news-datensaetze-inhalt} sind verschiedene deutsche Fake-News-Datensätze gelistet.
Die Eignung bezieht sich darauf, wie sinnvoll die Nutzung dieser jeweiligen Datensätze für das Trainieren eines Modells zum Erkennen von Fake News ist.

\begin{table}[ht]
    \centering
    \renewcommand{\arraystretch}{1.3}
    \begin{tabular}{|p{1.9cm}|p{2.3cm}|p{1.3cm}|p{1.3cm}|p{3.3cm}|p{1.5cm}|}
        \hline
        \rowcolor{lightgray} \textbf{Datensatz} & \textbf{Quelle} & \textbf{Anzahl Zeilen} & \textbf{Anteil Fake (\%)} & \textbf{Besonderheiten / Einschränkungen} & \textbf{Eignung} \\
        \hline
        \textbf{Fake News Dataset German} & University of Applied Sciences Upper Austria & 63.868 & 7{,}24\,\% & Geringer Anteil an Fake-News $\rightarrow$ unausgewogene Klassenverteilung; veraltet & Weniger geeignet \\
        \hline
        \textbf{German-Fake NC} & Fraunhofer-Institut für Sichere Informationstechnologie SIT & 489 & 100\,\% & Enthält nur Referenzen, keine Texte $\rightarrow$ keine direkte Textauswertung möglich; veraltet & Nicht geeignet \\
        \hline
        \textbf{FANG-COVID} & Association for Computational Linguistics & 41.242 & 31{,}97\,\% & Ausgewogene Klassen, vollständige Texte, viele Metadaten; veraltet & \textbf{Sehr geeignet} \\
        \hline
        \textbf{DeFaktS} & FZI Forschungszentrum Informatik & -- & -- & Kein Zugang $\rightarrow$ Nutzung ausgeschlossen & Nicht verfügbar \\
        \hline
    \end{tabular}
    \caption{Vergleich deutscher Fake-News-Datensätze hinsichtlich Umfang, Quellenlage und praktischer Eignung}
    \label{tab:deutsche-fake-news-datensaetze}
\end{table}

\begin{table}[ht]
    \centering
    \renewcommand{\arraystretch}{1.3}
    \begin{tabular}{|p{3cm}|p{2cm}|p{8cm}|}
        \hline
        \rowcolor{lightgray} \textbf{Datensatz} & \textbf{Kontext} & \textbf{Relevante Features} \\
        \hline
        \textbf{Fake News Dataset German} & nicht spezifisch & 'url', 'Titel', 'Body', 'Kategorie', 'Datum', 'Quelle', 'Fake', 'Art' \\
        \hline
        \textbf{German-Fake NC} & nicht spezifisch & 'Date', 'URL', 'False Statement 1 Location', 'False Statement 1 Index', 'Ratio of Fake Statements', 'Overall Rating' \\
        \hline
        \textbf{FANG-COVID} & COVID-19 Pandemie & 'article', 'date', 'header', 'label', 'url' \\
        \hline
        \textbf{DeFaktS} & -- &  -- \\
        \hline
    \end{tabular}
    \caption{Vergleich deutscher Fake-News-Datensätze hinsichtlich Kontext und relevanter Features}
    \label{tab:deutsche-fake-news-datensaetze-inhalt}
\end{table}

\section{Nutzung englischer Datensätze} \label{sec:english_datasets}

Eine in \cite{Simone2022} angewandte Lösung ist die Erstellung eines großen Datensatzes mit englischen Artikeln. Das Modell wurde entsprechend auf Englisch 
trainiert und die deutschen Artikel in der späteren Anwendung vor der Vorhersage übersetzt.
Fake News haben oft stilistische, semantische oder rhetorisch manipulierende Muster, welche sich je nach Sprache unterscheiden können.
Selbst mit modernen Transformern können diese Muster nicht sicher mit übersetzt werden, da diese einen übersetzen Artikel einer anderen Domäne zuordnen könnten
als den ursprünglichen Artikel \cite{hong2023disentanglingstructurestylepolitical}. 
Ein typisches Beispiel ist der Satz: „Natürlich wird das RKI bald neue Lockdowns empfehlen – die haben ja auch sonst nichts zu tun.“ 
Die Übersetzung ins Englische („Of course, the RKI will soon recommend new lockdowns – they have nothing else to do anyway“) wirkt sprachlich korrekt, 
verliert jedoch die ironisch-sarkastische Eigenschaft.
Das Ergebnis ist eine eher sachliche Aussage, welche vom englischen Modell sehr wahrscheinlich nicht mehr als Fake-News klassifiziert wird.
Solche stilistischen Abschwächungen sind ein Merkmal von maschinellen Übersetzungen, was sie in diesem Fall problematisch
für den Einsatz in der Fake-News-Erkennung macht \cite{kuehn2024enhancingrhetoricalfigureannotation}.

\section{Auswahl und Begründung des finalen Datensatzes}
\label{sec:auswahl_und_begruendung_finaler_datensatz}

Da der DeFaktS Datensatz nicht öffentlich verfügbar ist fällt dieser aus der Auswahl. 
Um ein Modell zu trainierien werden viele Daten benötigt. Die in Kapitel \ref{sec:bert_lightgbm} vorgestellten Datensätze umfassen von 20.000 bis hin zu Millionen Artikeln.
Aufrund der Größe entfällt der GermanFakeNC Datensatz.

Ein Problem, dass der Fake News Dataset German Datensatz (FNDG) bringt, ist die stark unausgewogene Klassenverteilung. Nur 7,24\% der 64.868 Einträge sind 
als Fake-News gekennzeichnet. Dadurch kann es beim Trainieren eines Modells dazu kommen, dass es überwiegend die häufiger vorkommenen echten Artikel 
lernt und Fake-News kaum oder gar nicht erkennt. 
Ein Modell könnte beispielsweise stets „echt“ vorhersagen und damit auf diesem Datensatz eine hohe Accuracy erreichen.
Bei späterer Anwendung würde es aber die entscheidenden Fälle nicht mehr differenzieren können. 
Das führt zu einem guten Accuracy-Wert, während Recall und F1-Score für die Fake-Klasse deutlich leiden. 
Das Modell verfehlt somit genau die Einträge, die für die Anwendung zentral sind.
Inhaltlich sind die Einträge aber auf einem breiten Gebiet verteilt, was einen Teil dieses Datensatzes für die Auswahl interessant macht.

Der verbleibende FANG-COVID Datensatz \cite{mattern-etal-2021-fang} fällt aufgrund seiner vielen Einträge und guter Klassenausgewogenheit
in die engere Auswahl. Problematisch bei diesem Datensatz ist aber, dass sich alle Einträge im COVID-19 Pandemie Kontext befinden.
Es besteht die Gefahr einer thematischen Überanpassung. Das Modell lernt vor allem, typische Begriffe, Erzählmuster und 
Formulierungen im Zusammenhang mit Pandemie-Fake-News zu erkennen. Dazu können zum Beispiel Impfgegner-Rhetorik, Verschwörungen oder auch Begriffe 
wie „PCR-Test“, „Lockdown“ oder „RKI“ gehören. In anderen Themenbereichen wie Politik, Migration oder Klima erkennt es dagegen manipulierte Inhalte 
unter Umständen nicht, da es diese Muster nie gelernt hat.
Außerdem kann das Modell semantische Verzerrungen entwickeln. Wörter, die in Pandemie-Fake-News häufig auftreten, könnten fälschlich als 
Indiz für eine Fake Klassifierung gewertet werden, auch wenn sie in neutralem oder echtem Kontext auftreten \cite{chen2023, nan2022improvingfakenewsdetection}. 
Dadurch steigt die Gefahr von False Positives bei echten Nachrichten außerhalb des COVID-Kontexts.
Ein Modell, das nur mit COVID-bezogenen Daten trainiert wurde, kann inhaltlich und stilistisch stark eingeschränkt generalisieren und 
verfehlt damit das Ziel, Fake News themenübergreifend zuverlässig zu erkennen.

Durch die Analyse der Datensätze ergeben sich folgende Möglichkeiten für das Trainieren eines Modells zur Fake-News Erkennung:
\begin{itemize}
    \item Nur FANG-COVID mit allen 41.242 Einträgen, aber 31,97\% Klassenausgewogenheit
    \item FANG-COVID mit ca. 26.000 Einträgen aber dafür 50/50 Klassenausgewogenheit
    \item FANG-COVID und FNDG kombiniert für bessere Generalisierung (105.110 Einträge, aber 16,94\% Klassenausgewogenheit)
    \item FANG-COVID und alle Fake Artikel von FNDG kombiniert für bessere Klassenausgewogenheit und mehr Daten
    (45.866 Einträge mit 38.82\% Klassenausgewogenheit)
    \item Alle Fake Einträge aus FANG-COVID und FNDG kombiniert (17.809 Fake Einträge) kombiniert mit Stichproben aus sowohl FANG-COVID als auch FNDG
    um eine sinnvolle Klassenausgewogenheit mit vielen Daten und einer guten Generalisierung zu erreichen (45.866 Einträge mit 38.82\% Klassenausgewogenheit,
    aber 28.057 echten Einträgen zu 50/50 aus FANG-COVID und FNDG).
\end{itemize}