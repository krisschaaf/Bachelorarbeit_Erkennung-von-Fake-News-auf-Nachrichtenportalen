\chapter{Natural Language Processing}
\label{chap:nlp}

\section{Machine Learning}
\label{sec:ml}

\subsection{Textbereinigung und Vorverarbeitung}
\label{sec:text_vorverarbeitung}

\begin{itemize}
    \item \textbf{Titel und Inhalt der Artikel zusammenfügen \cite{Buddhadev2025}}: Damit keine wichtigen Informationen verloren gehen,
    werden Titel und Inhalt des Artikels zusammengefasst. Gerade der Titel kann durch z.B. Clickbait (siehe \ref{sec:wie_definieren_sich_fake_news})
    schnell Hinweise auf eventuelle Fake News geben.

    \item \textbf{Akzente und Sonderzeichen entfernen \cite{Buddhadev2025} \cite{sabir2025}}: Akzente führen dazu, dass Wörter wie „café“ und „cafe“ unterschiedlich behandelt werden, 
    obwohl sie semantisch gleich sind. Das Entfernen dieser erhöht die Generalisierung. Sonderzeichen stören einfache Tokenizer (z. B. bei Bag-of-Words), führen zu vielen seltenen Tokens zu 
    überdimensionierten Vektoren (siehe \ref{sec:bag_of_words}).

    \item \textbf{Alle Buchstaben zu Kleinbuchstaben konvertieren \cite{sabir2025} \cite{SUDHAKAR2024101028}}: Ähnlich wie zum vorherigen Punkt
    erhöht die durchgehende Kleinschreibung aller Buchstaben die Generalisierung und verhindert somit unnötige Duplikate im Vokabular.

    \item \textbf{Leere Spalten entfernen \cite{SUDHAKAR2024101028}}: Leere Spalten enthalten keine Information. 
    Sie können bei der Vektorisierung oder Modellerstellung Fehler verursachen und werden als einfache Datenbereinigungsmaßnahme entfernt.

    \item \textbf{Kontraktionen auflösen (ans -> an das) \cite{Buddhadev2025}}: Im deutschen sind Kontradiktionen zwar nicht so häufig wie im englischen,
    sie kommen aber trotzdem vor und sollten aufgelöst werden. Dies vermeidet fragmentierte Token und verbessert die Semantik und Trennbarkeit im Modell.

    \item \textbf{Stoppwörter entfernen \cite{Buddhadev2025} \cite{sabir2025}}: Wörter wie „der“, „ist“, „und“ tragen wenig zur inhaltlichen Differenzierung bei. 
    Das Entfernen dieser verbessert die semantische Gewichtung relevanter Begriffe \cite{sarkar2018nlpguide}.

    \item \textbf{Rechtschreibfehler korrigieren \cite{sabir2025}}: Tippfehler führen zu seltenen Tokens und stören die Generalisierung. 
    In offiziellen Artikeln sind zwar selten Rechtschreibfehler zu finden, aber falls vorhanden, hilft die Korrektur zur Verbesserung
    der Modellqualität.

    \item \textbf{Lemmatisieren \cite{Buddhadev2025} \cite{sabir2025}}: Bei der Lemmatisierung werden verschiedene Wortformen auf die Grundform zurückgeführt 
    („läuft“, „lief“, „laufen“ wird zu „laufen“). So erkennt das Modell gleiche Bedeutungen trotz grammatischer Variation.

    \item \textbf{Tokenisierung \cite{sabir2025}}: In der Tokenisierung werden die Texte in einzelne Wörter oder Einheiten (Tokens) zerlegt, die für Modelle verarbeitbar sind. 
    Dies ist eine Grundvoraussetzung für alle weiteren NLP-Schritte wie TF-IDF oder Word Embeddings.
\end{itemize}

\subsubsection{Nutzung einer duale Feature-Pipeline}

Ein Problem welches das Entfernen der Akzente und Sonderzeichen und das Konvertieren aller Buchstaben zu Kleinbuchstaben mit sich bringt
ist, dass viele wichtige Hinweise zum Erkennen von Fake News verloren gehen. Wie in Kapitel \ref{sec:potenzielle_indikatoren} beschrieben,
sind fortlaufende Großschreibung, übermäßige Nutzung von Satzzeichen und falsche Zeichensetzung am Satzende potenzielle Indikatoren für Fake News.

Eine duale Feature-Pipeline kann dieses Problem lösen. Implementiert wird eine „cleaned“ Version (z.B. für inhaltliche Bedeutung) mit 
standardisierten, inhaltlichen Features und eine „rohe“ Version (z.B. für Stilmerkmale) mit stilistischen, rohen Textfeatures.

So werden semantische und stilistische Hinweise genau so genutzt wie ein Mensch es beim Lesen macht.

\paragraph{Die Notwendigkeit der Stilmerkmale} ist aber diskutierbar. Die Datensätze werden ausschließlich aus Artikeln von offiziellen
Nachrichtenportalen zusammengesetzt. Diese schreiben meist sauber, ohne Caps-Lock oder auffällige Sonderzeichen. Stilistische Merkmale wie viele 
Ausrufezeichen, Emojis oder absichtliche Rechtschreibfehler kommen dort nicht vor – also sind sie in diesem Fall auch keine verlässlichen Fake-News-Signale.

\subsection{Merkmalextraktion}
\label{sec:merkmalextraktion}

\subsubsection{Bag-of-words}
\label{sec:bag_of_words}

\subsubsection{TF-IDF}


\subsection{Machine Learning Modelle}
\label{sec:ml_modelle}

\subsubsection{Naive Bayes}

\subsubsection{Support Vector Machines}

\subsubsection{Logistische Regression}


\section{Deep Learning}
\label{sec:deep_learning}

\subsection{Word Embeddings}
\label{sec:word_embeddings}

\subsubsection{Word2Vec}

\subsubsection{GloVe}

\subsubsection{BERT-Tokenisierung}


\subsection{Deep Learning-Modelle}
\label{sec:deep_learning_modelle}

\subsubsection{CNN}

\subsubsection{LSTM}

\subsubsection{Transformer: BERT}


\section{Hybride Modelle}
\label{sec:hybride_modelle}

\subsection{convolutional neural network (CNN) along with bi-directional LSTM}

\cite{Buddhadev2025}

% \section{Diskussion: White Box vs. Black Box-Ansätze zur Erklärbarkeit}
% \label{sec:diskussion_erklaerbarkeit}