\chapter{Einleitung}
\label{chap:einleitung}

\section{Hintergrund: Die zunehmende Verbreitung von Fake News und deren gesellschaftliche Auswirkungen}
\label{sec:hintergrund}

%Fake News als Begriff politischer Kampfbegriff aus Trump-Wahlkämpfen -> stattdessen "Desinformationen" \cite{buerker2022fakenews}

%Support Vector Machines sind die mit am meisten genutzten Kategorisierungs Algorithmen \cite{Sharma:2024} p12
% -> Hier gibt es eine Aufzählung an Algorithmen die für das ML Model genutzt werden könnten! (auch deep learning im späteren Teil)

\subsection{Wann enstanden Fake News}

Fake News sind ein allgegenwärtiges Problem, doch hatten Sie Ihren ersten Auftritt bereits 44BC im römischen Reich \cite{socsci9100185}.
Auch während des amerikanischen Bürgerkriegs 1779 wurden Sie als politischer Schachzug von Benjamin Franklin genutzt.
Dieser schickte einen Brief an Captain Samuel Gerrish und schrieb in diesem über Grausamkeiten der Briten und deren Verbündeten. 
Diese Informationen wurde so veranschaulicht, dass sie die öffentliche Meinung bewusst beeinflussen sollten \cite{Sharma:2024}.

Der eigentliche Begriff "Fake News" wurde erst viele Jahre später durch Donald Trump im amerikanischen Wahlkampf 2016 bekannt \cite{Ashish2024}. 

Unter anderem ist Fake News auch ein Teil von Propaganda \cite{buerker2022fakenews}, welche schon lange als Mittel zur Meinungsmanipulation eines Volkes genutzt wird.

Heute ist Fake News die größte Drohung zu unserer angeblich freien Presse \cite{Sharma:2024}.

\subsection{Wie definieren sich Fake News und wie sind sie aufgebaut}

Fake News sind bewusst erstellte Online-Falschmeldungen, die teilweise oder vollständig unwahre Inhalte verbreiten, 
um Leser*innen gezielt zu täuschen oder zu manipulieren. Sie imitieren klassische Nachrichtenformate, nutzen auffällige Titel, emotionale Bilder und strategisch gestaltete
Inhalte, um Glaubwürdigkeit zu erzeugen und Aufmerksamkeit zu gewinnen. Ziel ist es, durch das Verbreiten dieser Inhalte Klicks, Reichweite und damit finanzielle oder ideologische Vorteile zu
erzielen. [Baptista2020].

Fake News fallen in die Kategorien Satire, Clickbait, Gerüchte, Stance News, Propaganda und Large Scale Hoaxes \cite{Sharma:2024}.

\begin{itemize}
    \item \textbf{Satire}: ist eine humorvolle oder übertriebene Darstellung gesellschaftlicher oder politischer Themen, die Kritik üben soll.
    \item \textbf{Clickbait}: bezeichnet reißerische Überschriften oder Vorschaubilder, die Neugier wecken und zum Anklicken eines Inhalts verleiten sollen, oft ohne den Erwartungen gerecht zu werden.
    \item \textbf{Gerüchte}: sind unbestätigte Informationen, die sich schnell verbreiten und oft falsch oder irreführend sind.
    \item \textbf{Stance News}: sind Nachrichten, die eine klare Meinung oder politische Haltung einnehmen, statt neutral zu berichten.
    \item \textbf{Propaganda}: ist die gezielte Verbreitung von Informationen oder Meinungen, um das Denken und Handeln von Menschen zu beeinflussen, meist im Interesse einer bestimmten Gruppe oder Ideologie.
    \item \textbf{Large Scale Hoaxes}: sind absichtlich erfundene Falschmeldungen oder Täuschungen, die weit verbreitet werden und viele Menschen täuschen sollen.
\end{itemize}

Die eigentliche Nachricht ist aufgebaut in folgende Teile:

\begin{itemize}
    \item \textbf{Quelle}: gibt den Ersteller der Nachricht an.
    \item \textbf{Titel}: erzielt die Aufmerksamkeit der Lesenden.
    \item \textbf{Text}: enthält die eigentliche Information der Nachricht.
    \item \textbf{Medien}: in Form von Bildern oder Videos.
\end{itemize}

Fake News können die Form von Text, Fotos, Filmen oder Audio annehmen und sind dementsprechend auf jeder Platform auffindbar, 
die die Verbreitung nicht unterbindet. Die momentan populärste Platform zum Teilen der Fake News ist WhatsApp \cite{Ashish2024}.

\subsection{Aus welcher Motivation entstehen Fake News}

Das Hauptinteresse der Ersteller der Fake News ist das Verdienen von Geld. Auf die Artikel wird Werbung geschaltet und 
anhand einer entsprechenden Reichweite ergibt sich der verdiente Betrag. Je mehr Reichweite, desto mehr Verdienst für die Ersteller \cite{socsci9100185}.

\subsection{Warum verbreiten sich Fake News}

In sozialen Medien neigen Nutzer aufgrund von FOMO (Fear of Missing Out) dazu, Fake News zu teilen, um Anerkennung zu gewinnen und soziale Zugehörigkeit zu erfahren. 
Besonders häufig werden kontroverse, überraschende oder bizarre Inhalte verbreitet – insbesondere dann, wenn sie starke Emotionen wie Freude, Wut oder Aufregung hervorrufen. 
Das Teilen solcher Inhalte stärkt das eigene Ansehen, da es signalisiert, über neue und relevante Informationen zu verfügen. 
Fake News bestehen meist aus eindrucksvoll präsentierten Falschinformationen. \cite{socsci9100185}

\subsection{Probleme beim Erkennen von Fake News}

Fake News können erst erkannt werden, nachdem diese erstellt und im Internet verbreitet wurden. \cite{Sharma:2024}

\subsection{Klassifizierungen}

Linguistische Features werden von Textmaterial auf verschiedenen Leveln gesammelt, z.B. Buchstaben, Wörter, Sätze 
und Features auf dem Satzlevel (Häufigkeit von Funktionswörtern? und Sätzen) \cite{secrypt17}

Text Tokenisierung \cite{Wagner:2010aa}

Aufteilung der Features in Syntactic, Semantic, Sentiment, Lexical, Style-based \cite{Sharma:2024} p7

\section{Wahl der Nachrichtenportale}
\label{sec:wahl_nachrichtenportale}

\section{Zielsetzung: Entwicklung einer Software zur automatisierten Fake-News-Erkennung}
\label{sec:zielsetzung}

\section{Aufbau der Arbeit}
\label{sec:aufbau}

