\chapter{Einleitung}
\label{chap:einleitung}

\section{Hintergrund: Die zunehmende Verbreitung von Fake News und deren gesellschaftliche Auswirkungen}
\label{sec:hintergrund}

\subsection{Wann entstanden Fake News}

Fake News sind ein allgegenwärtiges Problem, doch hatten Sie Ihren ersten Auftritt bereits 44BC im römischen Reich \cite{socsci9100185}.
Auch während des amerikanischen Bürgerkriegs 1779 wurden Sie als politischer Schachzug von Benjamin Franklin genutzt.
Dieser schickte einen Brief an Captain Samuel Gerrish und schrieb in diesem über Grausamkeiten der Briten und deren Verbündeten. 
Diese Informationen wurde so veranschaulicht, dass sie die öffentliche Meinung bewusst beeinflussen sollten \cite{Sharma:2024}.

Der eigentliche Begriff "Fake News" wurde erst viele Jahre später durch Donald Trump im amerikanischen Wahlkampf 2016 bekannt \cite{Ashish2024} 
und diente hierbei als politischer Kampfbegriff \cite{buerker2022fakenews}.

Unter anderem ist Fake News auch ein Teil von Propaganda \cite{buerker2022fakenews}, welche schon lange als Mittel zur Meinungsmanipulation eines Volkes genutzt wird.

Heute ist Fake News die größte Drohung zu unserer angeblich freien Presse \cite{Sharma:2024}.

\subsection{Wie definieren sich Fake News und wie sind sie aufgebaut}
\label{sec:wie_definieren_sich_fake_news}

Fake News sind bewusst erstellte Online-Falschmeldungen, die teilweise oder vollständig unwahre Inhalte verbreiten, 
um Leser*innen gezielt zu täuschen oder zu manipulieren. Sie imitieren klassische Nachrichtenformate, nutzen auffällige Titel, emotionale Bilder und strategisch gestaltete
Inhalte, um Glaubwürdigkeit zu erzeugen und Aufmerksamkeit zu gewinnen. Ziel ist es, durch das Verbreiten dieser Inhalte Klicks, Reichweite und damit finanzielle oder ideologische Vorteile zu
erzielen \cite{socsci9100185}.

Fake News fallen in die Kategorien Satire, Clickbait, Gerüchte, Stance News, Propaganda und Large Scale Hoaxes \cite{Sharma:2024}.

\begin{itemize}
    \item \textbf{Satire}: ist eine humorvolle oder übertriebene Darstellung gesellschaftlicher oder politischer Themen, die Kritik üben soll.
    \item \textbf{Clickbait}: bezeichnet reißerische Überschriften oder Vorschaubilder, die Neugier wecken und zum Anklicken eines Inhalts verleiten sollen, oft ohne den Erwartungen gerecht zu werden.
    \item \textbf{Gerüchte}: sind unbestätigte Informationen, die sich schnell verbreiten und oft falsch oder irreführend sind.
    \item \textbf{Stance News}: sind Nachrichten, die eine klare Meinung oder politische Haltung einnehmen, statt neutral zu berichten.
    \item \textbf{Propaganda}: ist die gezielte Verbreitung von Informationen oder Meinungen, um das Denken und Handeln von Menschen zu beeinflussen, meist im Interesse einer bestimmten Gruppe oder Ideologie.
    \item \textbf{Large Scale Hoaxes}: sind absichtlich erfundene Falschmeldungen oder Täuschungen, die weit verbreitet werden und viele Menschen täuschen sollen.
\end{itemize}

Die eigentliche Nachricht ist aufgebaut in folgende Teile:

\begin{itemize}
    \item \textbf{Quelle}: gibt den Ersteller der Nachricht an.
    \item \textbf{Titel}: erzielt die Aufmerksamkeit der Lesenden.
    \item \textbf{Text}: enthält die eigentliche Information der Nachricht.
    \item \textbf{Medien}: in Form von Bildern oder Videos.
\end{itemize}

Fake News können die Form von Text, Fotos, Filmen oder Audio annehmen und sind dementsprechend auf jeder Platform auffindbar, 
die die Verbreitung nicht unterbindet. Die 2024 populärste Platform zum Teilen der Fake News ist WhatsApp \cite{Ashish2024}.

\subsection{Aus welcher Motivation entstehen Fake News}

Das Hauptinteresse der Ersteller der Fake News ist das Verdienen von Geld. Auf die Artikel wird Werbung geschaltet und 
anhand einer entsprechenden Reichweite ergibt sich der verdiente Betrag. Je mehr Reichweite, desto mehr Verdienst für die Ersteller \cite{socsci9100185}.

\subsection{Warum verbreiten sich Fake News}

In sozialen Medien neigen Nutzer aufgrund von FOMO (Fear of Missing Out) dazu, Fake News zu teilen, um Anerkennung zu gewinnen und soziale Zugehörigkeit zu erfahren. 
Besonders häufig werden kontroverse, überraschende oder bizarre Inhalte verbreitet – insbesondere dann, wenn sie starke Emotionen wie Freude, Wut oder Aufregung hervorrufen. 
Das Teilen solcher Inhalte stärkt das eigene Ansehen, da es signalisiert, über neue und relevante Informationen zu verfügen. 
Fake News bestehen meist aus eindrucksvoll präsentierten Falschinformationen \cite{socsci9100185}.

Ein Grund für die schnelle Verbreitung von Fake News liegt in ihrer Aufmachung: Häufig wird die zentrale Aussage bereits in der Überschrift formuliert, 
oft mit Bezug auf konkrete Personen oder Ereignisse. Dadurch überspringen viele Leser den Artikel selbst, was die Wirkung von Schlagzeilen verstärkt. 
Die Inhalte sind meist kurz, wiederholend und wenig informativ. Anders als bei seriösen Nachrichten, bei denen Argumente überzeugen sollen, wirken Fake News über einfache Denkabkürzungen (Heuristiken) und die Bestätigung bestehender Überzeugungen. 
Nutzer müssen sich also nicht mit komplexen Inhalten auseinandersetzen, sondern lassen sich durch intuitive Übereinstimmungen überzeugen. 
Besonders bei geringer kognitiver Anstrengung – etwa durch Müdigkeit oder Unaufmerksamkeit – steigt die Wahrscheinlichkeit, dass Fake News geglaubt und weiterverbreitet werden \cite{horne2017}.

\subsection{Wer konsumiert Fake News}

Laut \cite{horne2017} sind folgende Gruppen die größten Konsumenten:

\begin{itemize}
    \item \textbf{Geringe Bildung oder digitale Kompetenz:} Personen mit niedriger formaler Bildung oder unzureichenden digitalen Fähigkeiten sind anfälliger für Falschinformationen.
    
    \item \textbf{Aussagen oder Nähe zur Informationsquelle:} Informationen von Personen, denen man persönlich nahe steht oder vertraut, werden eher geglaubt – unabhängig vom Wahrheitsgehalt.
    
    \item \textbf{Parteizugehörigkeit oder politische Überzeugung:} Menschen neigen dazu, Fake News zu glauben und zu verbreiten, wenn diese mit ihrer ideologischen Einstellung übereinstimmen.
    
    \item \textbf{Misstrauen gegenüber den Medien:} Wer etablierten Medien nicht vertraut, ist eher bereit, alternative (oftmals falsche) Quellen zu konsumieren und zu verbreiten.
    
    \item \textbf{Geringere kognitive Fähigkeiten:} Personen mit niedrigerer kognitiver Verarbeitungskapazität sind anfälliger für einfache, irreführende Inhalte und hinterfragen diese seltener kritisch.
\end{itemize}

Außerdem scheinen konservative, rechtsgerichtete Menschen, ältere Personen und weniger gebildete Menschen eher dazu zu neigen, Fake News zu glauben und zu verbreiten \cite{socsci9100185}.

\subsection{Welche potenziellen Indikatoren zum Erkennen bei Fake News gibt es}
\label{sec:potenzielle_indikatoren}

Das Erkennen von Fake News ist gerade deshalb problematisch, da diese erst erkannt werden können, nachdem sie erstellt und im Internet verbreitet wurden. \cite{Sharma:2024}

Gerade im Bereich der sozialen Medien gibt es aber relativ zuverlässige Indikatoren, die Fake News nach der Erstellung als solche zu enttarnen \cite{Hartwig2021}:

\begin{itemize}
    \item \textbf{Fortlaufende Großschreibung:} Beispiel: \texttt{GROßSCHREIBUNG}
    
    \item \textbf{Übermäßige Nutzung von Satzzeichen:} Beispiel: \texttt{!!!}
    
    \item \textbf{Falsche Zeichensetzung am Satzende:} Beispiel: \texttt{!!1}
    
    \item \textbf{Übermäßige Nutzung von Emoticons, besonders auffälliger Emoticons}
    
    \item \textbf{Nutzung des Standard-Profilbildes}
    
    \item \textbf{Fehlende Account-Verifizierung, besonders bei prominenten Personen}
\end{itemize}

Fake News in offiziellen Nachrichtenportalen zu erkennen, ist dagegen deutlich schwieriger.
Die aufgezählten stilistischen Mittel wie zum Beispiel die fortlaufende Großschreibung sind eher untypisch.
Stattdessen muss über die inhaltliche Bedeutung erkannt werden ob die Artikel wahr oder falsch sind.

\section{Zielsetzung: Entwicklung einer Software zur automatisierten Fake-News-Erkennung}
\label{sec:zielsetzung}

Motiviert durch die Arbeiten der University of Applied Sciences Upper Austria \cite{Simone2022} und der TU Darmstadt \cite{Hartwig2021} wird in 
dieser Arbeit die Entwicklung eines weiteren Tools dokumentieren.
Dieses Tool soll wie auch das Browser Plugin TrustyTweet eine Unterstützung zum Erkennen von Fake News anbieten.
Ob dieses Tool auch als Browser Plugin oder als eine andere Form der Software implementiert wird, steht zum jetzigen Zeitpunkt noch nicht fest.
Auch ob eine Black Box oder White Box Architektur genutzt wird, - das heißt, kann der User zum Beispiel sehen, warum der Artikel als Fake News 
deklariert wird - wird im Laufe der Arbeit entschieden.
Ziel ist es, dass das Tool nicht wie TrustyTweet auf Twitter eingesetzt wird, sondern auf verschiedenen Nachrichtenportalen.
Welche Portale hierfür genutzt werden, hängt von der noch ausstehenden Entscheidung der genutzten Datensätze ab.

\section{Wahl der Nachrichtenportale}
\label{sec:wahl_nachrichtenportale}

Im Paper der University of Applied Sciences Upper Austria \cite{Simone2022} wird die Qualität verschiedener deutscher Nachrichtenportale mit 
„Machine Learning“-Modellen getestet. 
Das Ergebnis zeigt, dass Spiegel, Die Zeit and Süddeutsche die besten Portale sind und Express, BZ-Berlin and Bild die schlechtesten - 
also auch am meisten Fake News verbreiten.
Als populärstes der schlechtesten Portale wird Bild eines der zu analysierenden Portale. Dieses befindet sich politisch im rechten Spektrum.
Als Gegensatz wird außerdem TAZ analysiert um einen Vergleich zu einem bekannten linken Nachrichtenportal zu haben.
Das dritte Portal %TODO:

\section{Aufbau der Arbeit}
\label{sec:aufbau}



