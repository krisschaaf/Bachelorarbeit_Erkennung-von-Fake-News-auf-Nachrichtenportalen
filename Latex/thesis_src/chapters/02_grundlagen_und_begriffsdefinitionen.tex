\chapter{Grundlagen und Begriffsdefinitionen}
\label{chap:grundlagen_und_begriffsdefinitionen}

\section{Definition „Fake News“: Merkmale, Ziele, Beispiele}
\label{sec:definition_fake_news}


\subsection{Klassifizierungen}

Linguistische Features werden von Textmaterial auf verschiedenen Leveln gesammelt, z.B. Buchstaben, Wörter, Sätze 
und Features auf dem Satzlevel (Häufigkeit von Funktionswörtern? und Sätzen) \cite{secrypt17}

Text Tokenisierung \cite{Wagner:2010aa}

Aufteilung der Features in Syntactic, Semantic, Sentiment, Lexical, Style-based \cite{Sharma:2024} p7

\section{Kategorisierung der Fake News Detection-Ansätze}
\label{sec:kategorisierung_fake_news_detection_ansätze}

\section{Warum der Fokus auf Machine Learning?}
\label{sec:warum_fokus_machine_learning}

\section{Überblick über relevante Plattformen und deren Rolle im Medienkonsum}
\label{sec:plattformen_und_medienkonsum}

