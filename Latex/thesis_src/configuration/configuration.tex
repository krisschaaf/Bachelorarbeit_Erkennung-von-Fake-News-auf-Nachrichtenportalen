% !TEX root = ../thesis.tex
%
% configurations
%

% English Language support
% -> uncomment if needed
% Beta!
%\fullenglish{yes}
\fullenglish{no}

% text field
%-> replace supervisor names with correct ones
\firstSupervisor{Prof. Dr. Stefan Sarstedt}
\secondSupervisor{Prof. Dr. Marina Tropmann-Frick}

% text field
%-> replace title with your thesis title
\thesisTitle{Entwicklung einer Software zur Erkennung von Fake News auf Nachrichtenportalen}
\thesisTitleEN{Development of a software for the detection of fake news on news portals}

% text field
%-> replace the key words with your own key words
\keywordsDE{Machinelles Lernen, Fake News Erkennung, Textklassifikation, Natural Language Processing, Transformer, BERT, RoBERTa, LightGBM, Chrome-Extension} 
\keywordsEN{machine learning, fake news detection, text classification, natural language processing, transformer, BERT, RoBERTa, LightGBM, Chrome extension}

% text field
\abstractDE{Die Bachelorarbeit dokumentiert die Entwicklung einer Software zur Erkennung von Fake News auf Nachrichtenportalen mithilfe von Methoden des 
maschinellen Lernens und modernen NLP-Ansätzen. Ziel ist die automatisierte Klassifikation von Nachrichtenartikeln als echt oder gefälscht auf 
Basis semantischer und stilistischer Merkmale. In dieser Arbeit erfolgt ein Fine-Tuning verschiedener Transformer-Modelle, deren Embeddings anschließend als 
Eingabe für LightGBM-Modelle dienen, um die Klassifikation von Fake News effizient umzusetzen. Die Lösung wird für drei politisch diverse deutsche 
Nachrichtenportale (BILD, taz, Der Spiegel) implementiert. Evaluiert werden die Ergebnisse auf Basis eines eigens zusammengestellten Datensatzes. 
Die Arbeit erläutert Vorverarbeitungsschritte, Modellarchitekturen, Trainingsverfahren und zeigt durch systematische 
Evaluation die Effektivität der entwickelten hybriden Klassifikationslösung auf.\dots} 

\abstractEN{The bachelor's thesis documents the development of software for detecting fake news on news portals using machine learning methods and
modern NLP approaches. The goal is the automated classification of news articles as real or fake based on semantic and stylistic features. 
In this work, various transformer models are fine-tuned, and their embeddings are subsequently used as input for LightGBM models to efficiently implement
fake news classification. The solution is implemented for three politically diverse German news portals (BILD, taz, Der Spiegel). 
The results are evaluated using a custom-compiled dataset. The thesis explains preprocessing steps, model architectures, and training procedures 
and demonstrates the effectiveness of the developed hybrid classification solution through systematic evaluation.\dots}

% text field
%-> replace john with your name
\thesisAuthor{Kristoffer Schaaf}

% text field
%-> enter the submission date
\submissionDate{03.07.2025} %TODO

% switch - uncomment only one
%-> uncomment NDA or public
%\NDA{yes}
\NDA{no}

% switch - uncomment only one
%-> uncomment old standard cover or cover Corporate Design 2017
\Cover{CD2017}
%\Cover{CD2017NoLogo}
%\Cover{Std2018}
%\Cover{Std2018_green} 			% with green bar

% switch - uncomment only one
%-> uncomment to show list of figures or not
\ListOfFigures{yes}
%\ListOfFigures{no}

% switch - uncomment only one
%-> uncomment to show list of tables or not
\ListOfTables{yes}
%\ListOfTables{no}

% switch - uncomment only one
%-> uncomment to show list of accronyms or not
\ListOfAccronyms{yes}
%\ListOfAccronyms{no}

% switch - uncomment only one
%-> uncomment to show list of symbols or not
\ListOfSymbols{yes}
%\ListOfSymbols{no}

% switch - uncomment only one
%-> uncomment to show list of glossary entries or not
\Glossary{yes}
%\Glossary{no}

% switch - uncomment only one
%-> uncomment the study course you are in
%\studycourse{ITS}
%\studycourse{TI}
\studycourse{AI}
%\studycourse{WI}
%\studycourse{EI}
%\studycourse{REE}
%\studycourse{BMP}		
%\studycourse{BMP-hp}	 % Internship Report in M&P
%\studycourse{BMT}
%\studycourse{BMT-st}    % Study / home assignment in BMT
%\studycourse{BMT-hp}    % Internship Report in BMT
%\studycourse{MI}
%\studycourse{MIK}
%\studycourse{MA}
